\documentclass{beamer}
\usepackage{booktabs}
\usepackage{xcolor}
\usepackage{changepage}
\usepackage[export]{adjustbox}
\usepackage{booktabs}

\usetheme[numbering=fraction,progressbar=frametitle,sectionpage=none]{metropolis}


% Backup
\newcommand{\backupbegin}{%
   \newcounter{finalframe}
   \setcounter{finalframe}{\value{framenumber}}
}
\newcommand{\backupend}{%
   \setcounter{framenumber}{\value{finalframe}}
}


% Colors
\definecolor{myBlue}{RGB}{21, 56, 110}
\definecolor{myRed}{RGB}{174,0,34}
\definecolor{myRedBg}{RGB}{251, 217, 224}
\definecolor{greySAV}{RGB}{167,166,166}
\definecolor{greyCU}{RGB}{44,46,53}

\setbeamercolor{title}{fg=myBlue}
\setbeamercolor{background canvas}{bg=white}
\setbeamercolor{normal text}{fg=black}
\setbeamercolor{frametitle}{fg=myBlue, bg=white}
\setbeamercolor{section title}{fg=myBlue}
\setbeamercolor{title separator}{fg=myRed,bg=myRedBg}
\setbeamercolor{progress bar}{fg=myRed,bg=myRedBg}
\setbeamercolor{structure}{fg=myRed}


% Thicker progress bar
\makeatletter
\setlength{\metropolis@titleseparator@linewidth}{1pt}
\setlength{\metropolis@progressonsectionpage@linewidth}{1pt}
\setlength{\metropolis@progressinheadfoot@linewidth}{1pt}
\makeatother


% Commands
\newcommand{\bluetext}[1]{%
  \textcolor{myBlue}{#1}
}
\newcommand{\redtext}[1]{%
  \textcolor{myRed}{#1}
}


% Variables
\def\pt{\ensuremath{p_\mathrm{T}}}


% Title
\title[FCCcalo]{Noble Liquid Calorimetry for Future Circular Collider}
\author[Smiesko, Faltova]{Jana~Faltová\inst{1},
                          \underline{Juraj~Smieško}\inst{1,2}}
\institute[CU, SAS]{\inst{1} Charles University, Czechia \\
                    \inst{2} Slovak Academy of Sciences, Slovakia}
\date[2020-Dec-03]{\footnotesize UNCE Seminar, Prague \\
                   December 03, 2020} % \\
                   % (Ab urbe condita 2773)}


% ---------------------------------------------------------------------------- %
\begin{document}

{%
  \setbeamercolor{background canvas}{bg=greyCU}
  \begin{frame}[noframenumbering]
    \centering
    \vspace{1cm}
    \includegraphics[width=.25\textwidth]{figures/CU_red_white_logo.pdf}
    \thispagestyle{empty}
  \end{frame}
}

\begin{frame}
  \titlepage{}
  \thispagestyle{empty}
\end{frame}


% \begin{frame}
%   \frametitle{Overview}
%
%   \tableofcontents
% \end{frame}


% ---------------------------------------------------------------------------- %
\section{Future Circular Collider}

\begin{frame}
  \frametitle{Future Circular Collider}

  \begin{columns}[c]
    \column{.6\textwidth}
    \includegraphics[width=\linewidth]{figures/FCC_CLIC.png}
    \vspace{-10mm}
    \tiny{Image: \href{https://alumni.cern/news/226282}{CERN}}

    \column{.4\textwidth}
    European Particle Physics Strategy:
    \begin{itemize}
      \item Investigate feasibility for FCC at CERN
      \item As global endeavor
      \item Ring circumference 100 km
      \item Electron-positron collider as first step
      \item Hadron collider: $\sqrt{s} = 100$~TeV
    \end{itemize}
  \end{columns}
\end{frame}

\begin{frame}
  \frametitle{FCC-ee: Lepton Collider}

  \includegraphics[width=.49\linewidth]{figures/FCC_ee_ring.png}
  \includegraphics[width=.49\linewidth]{figures/FCC_ee_operation_plan.png}

  FCC-ee characteristics:
  \begin{columns}[c]
    \column{.5\textwidth}
    \begin{itemize}
      \item For precision physics
      \item Higgs factory
      \item Electroweak precision $10^{-3} \rightarrow 10^{-5}$
    \end{itemize}

    \column{.5\textwidth}
    \begin{itemize}
      \item Operation at four energies
      \item Clean environment
      \item ``Continuous'' beams
    \end{itemize}
  \end{columns}
\end{frame}

\begin{frame}
  \frametitle{FCC-hh: Hadron Collider}

  \includegraphics[width=.49\linewidth]{figures/FCC_hh_ring.pdf}
  \includegraphics[width=.49\linewidth]{figures/FCC_hh_Zprime_ll.pdf}

  FCC-hh characteristics:

  \begin{columns}[c]
    \column{.5\textwidth}
    \begin{itemize}
      \item Continuation of energy frontier
    \end{itemize}

    \column{.5\textwidth}
    \begin{itemize}
      \item Operation at 100 TeV energies
    \end{itemize}
  \end{columns}
\end{frame}


% ---------------------------------------------------------------------------- %
\section{FCC Detectors}

\begin{frame}
  \frametitle{FCC-ee: CLD Detector}

  \begin{columns}[c]
    \column{.6\textwidth}
    \includegraphics[width=\linewidth]{figures/FCC_ee_CLD_text.png}

    \column{.4\textwidth}
    \begin{itemize}
      \item Based on the detector for CLIC
      \item Silicon vertex detector and tracker
      \item 3D-imaging highly-granular calorimeter
      \item Coil outside calorimeter system
      \item Proved concept, understood performance
    \end{itemize}
  \end{columns}
\end{frame}

\begin{frame}
  \frametitle{FCC-ee: IDEA Detector}

  \begin{columns}[c]
    \column{.6\textwidth}
    \includegraphics[width=\linewidth]{figures/FCC_ee_IDEA.png}

    \column{.4\textwidth}
    \begin{itemize}
      \item New, innovative, possibly more cost-effective design
      \item Silicon vertex detector
      \item Short-drift, ultra-light wire chamber
      \item Dual-readout calorimeter
      \item Thin and light solenoid coil inside calorimeter system
    \end{itemize}
  \end{columns}
\end{frame}

\begin{frame}
  \frametitle{FCC-hh: Reference Detector}

  \begin{adjustwidth}{-2em}{-2em}
    \includegraphics[width=\linewidth]{figures/FCC_hh_ref_detector.png}
  \end{adjustwidth}
\end{frame}

\begin{frame}
  \frametitle{FCC-hh: Reference Detector}

  \begin{adjustwidth}{-2em}{-2em}
    \includegraphics[width=\linewidth]{figures/FCC_hh_ref_detector_xsec.pdf}
  \end{adjustwidth}
\end{frame}


% ---------------------------------------------------------------------------- %
\section{FCC Calorimetry}

\begin{frame}
  \frametitle{FCC Calorimetry}

  \begin{columns}[c]
    \column{.5\textwidth}
    \bluetext{Requirements:}
    \begin{itemize}
      \item Jet-jet invariant mass resolution to resolve $W$ from $Z \sim 3$\%
      \item EM resolution 15\% sufficient to sustain jet resolution
            \begin{itemize}
              \item Precise identification of $\gamma$'s and $\pi_0$'s in
                    dense topologies
            \end{itemize}
      \item Crystal and LAr for good EM resolution
      \item CALICE and Dual Readout good jet resolution
    \end{itemize}

    \column{.5\textwidth}
    \bluetext{Energy resolution parametrization:}
    \[ \frac{\sigma_\text{E}}{\langle E\,\rangle} = \frac{a}{\sqrt{E\,}} \oplus
                                                    \frac{b}{E} \oplus c \]

    \bluetext{Typical values:}
    \begin{center}
      \small
      \begin{tabular}{lcc}
        Technology   & $a$~[\%] & $c$~[\%] \\
        \midrule
        CALICE       & 15       & 1 \\
        Dual Readout & 10       & 1 \\
        LAr          & 9        & --- \\
        Crystal      & 3--5     & 0.5 \\
      \end{tabular}
    \end{center}
  \end{columns}

  \vspace{2ex}
  \redtext{High granularity and Particle Flow needed to achieve energy resolution
           of 3\%}
\end{frame}


\begin{frame}
  \frametitle{Particle Flow}

  \includegraphics[width=\linewidth]{figures/particle_flow_diagram.pdf}
  \begin{adjustwidth}{-1em}{-1em}
    \vspace{-1.5ex}
    Composition: 30\% + 70\% \hspace{9.5em} 60\% + 30\% + 10\%
  \end{adjustwidth}

  \vspace{0.5em}
  \bluetext{Idea:}
  \begin{columns}[c]
    \column{.5\textwidth}
    \begin{itemize}
      \item Reconstruct every particle in the event with the best possible
            precision
      \item Combine the measurements in subdetectors in an optimal way
    \end{itemize}

    \column{.5\textwidth}
    \begin{itemize}
      \item Charged particles dominated by tracker
      \item Calorimetry mostly for neutral particles
      \item \bluetext{Enemy: Confusion}
    \end{itemize}
  \end{columns}

\end{frame}


\begin{frame}
  \frametitle{FCC-ee IDEA Calorimeter}

  \bluetext{Traits:}
  \begin{itemize}
    \item Dual readout calorimeter with 1.5~mm fiber pitch
          Cherenkov/Scintillation dual-readout
    \item Single EM + HAD sampling calorimeter
    \item No mechanical longitudinal segmentation ($\sim 7 \lambda_\text{I}$)
    \item Good EM intrinsic energy resolution
    \item Excellent hadronic resolution
  \end{itemize}

  \includegraphics[width=0.49\linewidth]{figures/IDEA_calorimeter_3dview.png}
  \includegraphics[width=0.49\linewidth]{figures/IDEA_calorimeter_fibers.png}

\end{frame}

\begin{frame}
  \frametitle{CALICE}

  Collaboration of mostly Si/Tungsten based high granularity calorimeters

  \begin{columns}[c]
    \column{.4\textwidth}

    \bluetext{Traits:}
    \begin{itemize}
      \item Large area silicon detectors
      \item Si Photomultipliers
      \item Highly integrated front-end electronics with timing
      \item Very large number of channel
    \end{itemize}

    \column{.6\textwidth}
    \includegraphics[width=\linewidth]{figures/CALICE_diagram.png}
  \end{columns}
\end{frame}

\begin{frame}
  \frametitle{FCC-ee CLD Calorimeter}

  \bluetext{CLD proposal:}
  \begin{itemize}
    \item 40 layers SiW ECAL (22 $X_0$)
    \item 60 layers Scint/Steel HCAL (7.5 $\lambda_\text{I}$ +
          1 $\lambda_\text{I}$ in ECAL)
  \end{itemize}

  \includegraphics[width=0.49\linewidth]{figures/CLD_jet_performance.pdf}
  \includegraphics[width=0.49\linewidth]{figures/CLD_photon_performance.png}
\end{frame}


% ---------------------------------------------------------------------------- %
\section{Noble Liquid (LAr) Calorimetery}

\begin{frame}
  \frametitle{Noble Liquid (LAr) Calorimetry}


  \begin{columns}[c]
    \column{.5\textwidth}
    \begin{itemize}
      \item Tested technology in ATLAS
      \item Less than $10\%/\sqrt{E}$ demonstrated
    \end{itemize}

    \bluetext{Improvements for FCC-ee}
    \begin{itemize}
      \item Higher granularity
            \begin{itemize}
              \item Transverse and in depth
              \item More signal traces
            \end{itemize}
      \item Improve EM resolution
            \begin{itemize}
              \item Higher sampling fraction $\rightarrow$ thicker detector
            \end{itemize}
      \item Low energy photons (300 MeV)
            \begin{itemize}
              \item Low mass cryostat
              \item Smaller noise term
            \end{itemize}
    \end{itemize}

    \column{.5\textwidth}
    \includegraphics[width=.8\linewidth]{figures/FCC_hh_LAr_electron_performance_eta.pdf}
    \includegraphics[width=.8\linewidth]{figures/FCC_hh_LAr_electron_performance_mu.pdf}
  \end{columns}
\end{frame}


% ---------------------------------------------------------------------------- %
\section{Conclusions and Plans}

\begin{frame}
  \frametitle{Conclusion and Plans}

  \begin{itemize}
    \item LAr calorimeters were investigated mostly for FCC-hh
    \item Investigate LAr performance in the FCC-ee framework
          \begin{itemize}
            \item Replace existing CLD/IDEA calorimeters with LAr
          \end{itemize}
    \item Implement particle flow algorithm for LAr calorimeter
  \end{itemize}
\end{frame}

% \appendix
% \backupbegin{}
%
% \begin{frame}
%   \frametitle{Backup}
% \end{frame}
%
% \backupend{}

\end{document}
